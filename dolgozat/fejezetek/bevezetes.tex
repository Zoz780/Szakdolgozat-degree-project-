\Chapter{Bevezetés}
\label{Chap:bevezetes}

Egy FPS (\textit{First-Person Shooter}) játék elkészítése különféle számítási- és optimalizálási problémákat vet fel. A számítógépes grafika és a játékfejlesztés segítségével lehetőség adódott a termelésinformatikához kapcsolódó optimalizálási problémák bemutatására.

Azon számítógépes játékoknak, amelyek célja a minél valószerűbb megjelenítés, számos olyan eleme van, amely optimalizálás nélkül olyan erőforrásigényes lenne, amely a manapság elérhető csúcskategóriás számítógépeken sem futna megfelelően. Ilyen elem például a karakterek mozgásterének definiálása, a lövés során eltalált objektumok metszéspontjának számítása, az ellenfelek viselkedésének szimulálása.

Megjelenítés szempontjából nagyon fontos, hogy a videokártyának megfelelő formátumban, egyben adjuk át a kirajzolni kívánt elemeket, mert így hatékonyabb lesz a megjelenítés, jobban ki lehet használni a rendelkezésre álló erőforrásokat, illetve a magasabb másodpercenkénti képkockaszámot.

A problémakör egyik fontos eleme az ellenfelek bejárható területének kezelése. Tulajdonképpen ez is egyfajta ütközésvizsgálatot jelent a falakra, objektumokra nézve. A dolgozatban bemutatásra kerülnek azon térpartícionálásos megoldások, amelyekkel a számítások gyorsabban, hatékonyabban elvégezhetők.

Az ütközésvizsgálat egy másik tipikus esete az ilyen jellegű játékokban a lövedékek útjának és metszéspontjainak kiszámítása. Ennél például jelentős optimalizációt érhetünk el, ha csak azon objektumokat vesszük figyelembe, amelyek a lövedék haladási irányába esnek.

% ÚJ RÉSZ

Az ellenfelek találatregisztrálásához az optimális teljesítmény érdekében, nem közvetlen a játékos számára látható, komplex geometriára végezzük el. Ezen egyszerűbb geometria részletességének meghatározása nagyon fontos tényező, mert ha túl egyszerű, nem lesz pontos a találatregisztrálás, ha túl komplex, akkor pedig túl nagy lesz az erőforrásigénye.

Fontos eleme egy játéknak az ellenfelek viselkedésének modellezése is. Minden ellenfélnek van egy elsődleges célpontja, viszont különböző környezeti tényezők befolyásolhatják azt. Fontos, hogy a játékos úgy érezze, hogy a mesterséges intelligencia ésszerűen cselekszik, ennek megvalósítását is részletezi a későbbiekben a dolgozat.
